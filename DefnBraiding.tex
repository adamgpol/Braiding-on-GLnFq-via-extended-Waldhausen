\subsection{Braided monoidal categories}
\label{DefnBraiding}
Conventionally one defines a braided monoidal category to be a monoidal category equipped with an additional natural isomorphism \[\gamma_{x,y}: \xymatrix{x \otimes y \ar[r]^{\sim} & y \otimes x}\] satisfying the so-called hexagon identities. 
In their foundational work on braided monoidal categories, Joyal and Street gave an equivalent way to define such structures \cite{Joyal-StreetBTC}. Namely, a braided monoidal category is equivalently a monoid object in the monoidal $2$-category $\mathbf{Mon}$. This $2$-category has objects monoidal categories, $1$-morphisms monoidal functors and $2$-morphisms monoidal natural transformation. The Cartesian product endows it with a monoidal structure. %We call such structures {\em doubly monoidal categories}
As this is the notion of braiding which we shall use throughout this paper we will now unpack the Joyal and Street definition.

\begin{Definition}
A braided monoidal category is a monoidal category \\$(\CCC, m,\monunit, \alpha,\lambda, \rho)$ endowed with the following additional data:

\begin{enumerate}
    \item a functor $\mu: \CCC \otimes \CCC \to \CCC$;
    \item an isomorphism $\epsilon: \monunit \to \mu(\monunit,\monunit)$;
    \item a natural isomorphism
    \begin{equation}
        \stik{1}{
                \CCC^{\otimes 4} \ar{r}{\mu\times\mu} \ar{d}{\overline{m}^2} \& \CCC^{\otimes 2} \ar{d}{m}  \ar[Rightarrow, shorten <=1em,shorten >=1em]{dl}{\beta} \\
                \CCC^{\otimes 2} \ar{r}{\mu} \& \CCC
                }
    \end{equation}
    where $\overline{m}^2$ is the induced monoidal structure on $\CCC \otimes \CCC$.
    \item natural isomorphisms $\varrho: \mu(-,\monunit) \Rightarrow \Id$ and $\Lambda: \mu(\monunit,-) \Rightarrow \Id$.
   
\end{enumerate}

 This data must make the following diagrams commute

Coherence with associator, i.e. the cube:
   
\newcommand{\TFlabel}{m^2\Id^2}
\newcommand{\TBklabel}{m\Id}
\newcommand{\TLlabel}{\mu^3}
\newcommand{\TRlabel}{\mu^2}
\newcommand{\BFlabel}{\overline{m}}
\newcommand{\BBklabel}{m}
\newcommand{\BLlabel}{\mu^2}
\newcommand{\BRlabel}{\mu}
\newcommand{\FRlabel}{\overline{m}}
\newcommand{\FLlabel}{\Id^2m^2}
\newcommand{\BkRlabel}{m}
\newcommand{\BkLlabel}{\Id m}

\begin{equation}%top-bottom,front-back,left-right
\label{braidingcube}
\stik{1.5}{
{} \& \CCC^{\otimes 3} \ar{rr}[name=topback,below]{\TBklabel} \ar{dd}[name=backleft,yshift=2.5ex]{\BkLlabel} \& {} \& \CCC^{\otimes 2} \ar{dd}[name=backright,right]{\BkRlabel}\\
\CCC^{\otimes 6} \ar[crossing over]{rr}[name=topfront,below,xshift=2em,yshift=-0.2ex]{\TFlabel}\ar{dd}[name=frontleft,left]{\FLlabel} \ar{ur}[name=topleft,above]{\TLlabel} \& {} \& \CCC^{\otimes 4} \ar{dd}[name=frontright,yshift=-2.5ex,xshift=0.1em]{\FRlabel} \ar{ur}[name=topright,below]{\TRlabel} \& {}\\
{} \& \CCC^{\otimes 2} \ar{rr}[name=bottomback,below,xshift=-2em]{\BBklabel} \& {} \& \CCC\\
\CCC^{\otimes 4} \ar{rr}[name=bottomfront,below]{\BFlabel} \ar{ur}[name=bottomleft,yshift=-0.2pc]{\BLlabel} \& {} \& \CCC^{\otimes 2} \ar{ur}[name=bottomright,below]{\BRlabel}
\arrow[Rightarrow,to path={(topright) to[bend left] (topback)}, shorten <=0.4em]{}{}
\arrow[Rightarrow,to path={(frontright) to[bend right] (bottomfront)}]{}{}
\arrow[Rightarrow,crossing over,to path={(frontright) to[bend right] (topright)}]{}{}
\arrow[Rightarrow,to path={(bottomfront) to[bend right] (bottomleft)}, shorten <= 0.1em,shorten >=0.5em]{}{}
\arrow[Rightarrow,to path={(bottomleft) to[bend left] (backleft)}]{}{}
\arrow[Rightarrow,to path={(topback) to[bend left] (backleft)}]{}{}
\latearrow{commutative diagrams/crossing over}{2-3}{4-3}{}
\latearrow{commutative diagrams/crossing over}{2-1}{2-3}{}
}
\end{equation}
where the diagram of two morphisms is
\[
\stik{0.8}{
{} \& \mu\circ\overline{m}\circ \Id^2 m^2 \rar{\beta\Id_{\Id^2 m^2}} \& m\circ\mu^2\circ \Id^2 m^2 \drar{m(\beta\Id_{\mu})}\\
\mu\circ\overline{m}\circ m^2\Id^2 \urar{\mu(\alpha\times\alpha)} \drar{\beta\Id_{m^2\Id^2}} \& {} \& {} \& m\circ\Id m\circ\mu^3\\
{} \& m\circ\mu^2\circ m^2\Id^2 \rar{m(\Id_{\mu}\beta)} \& m\circ m\Id\circ \mu^3 \urar{\alpha\Id_{\mu^3}}
}
\]
and coherence with units:
\[
\stik{1}{
                m  (\mu \times \monunit) \ar[Rightarrow]{r}{m (\mu \times \epsilon)} \ar[Rightarrow]{d}{\rho \mu} \& m \mu^2 (\Id^2 \times \monunit^2) \ar[Rightarrow]{d}{\beta (\Id^2 \times \monunit^2)}  \\
                \mu\ar[Leftarrow]{r}{\mu \overline{\rho}^2} \& \mu \overline{m}^2(\Id^2\times\monunit^2)
                } \]
                
\[
\stik{1}{
                m (\monunit \times \mu) \ar[Rightarrow]{r}{m (\epsilon \times \mu)} \ar[Rightarrow] {d}{\lambda \mu} \& m \mu^2 (\monunit^2\times\Id^2) \ar[Rightarrow]{d}{\beta(\monunit^2\times\Id^2)}   \\
                \mu \ar[Leftarrow]{r}{\mu \overline{\lambda}^2} \& \mu \overline{m}^2 (\monunit^2\times\Id^2)
                }\] 
                
\end{Definition}

%What about when C is linear? Should do everything in terms of \boxtimes and such. \lena{No, just say it carries over to this setting}

A monoidal category $(\CCC, \otimes,\monunit, \alpha,\lambda, \rho)$ with braiding $\gamma$ can be endowed with the above structure by defining $\beta = \Id\otimes \gamma \otimes \Id$, $\varrho = r^{-1}$ and $\Lambda = \lambda^{-1}$. Conversely, given the above structure on $\CCC$ one associates the braiding 
\begin{equation}
\label{ExtractBraiding}
\gamma = (\varrho \otimes \Lambda)^{-1} \circ \beta^{-1} \circ \mu(\rho,\lambda)^{-1} \circ \mu(\lambda,\rho) \circ \beta \circ (\Lambda \otimes \varrho). 
\end{equation}
These constructions extend to yield equivalences between the $2$-categories of braided monoidal categories and monoid objects in $\Mon$.


%\begin{Remark}
%The equivalence between these two definitions of a braided monoidal category is an instance of Dunn additivity for the $E_n$ operads \cite{Dunn}. The conventional definition says that a braided monoidal category is an algebra over the {\em parenthesized braid operad}, which is an $E_2$-operad in groupoids ***citation?***\lena{if true, citation is not needed}. The definition we use throughout this paper is that of an $E_1$-algebra in the category of $E_1$-algebras. The $n=2$ case of Dunn additivity, namely that $E_2$-alg$\ \simeq E_1$-alg$(E_1$-alg$)$, is the desired equivalence.
%\end{Remark}

