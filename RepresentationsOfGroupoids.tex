\section{Representations of groupoids}
\label{GroupoidRep}
In this section we recall some basic notions regarding representations of groupoids and give a convenient formula for the push forward functor.

\begin{Definition}
A \emph{representation} of a groupoid $G$ is a functor $G\to \Vect_{\field}$ which is non-zero only on finitely many connected components of $G$. Denote the category of representations of $G$ by $\Rep(G)$.
\end{Definition}

\begin{Definition}
\label{2Fiber}
Let $G\xrightarrow{p} H$ be a morphism of groupoids. Recall that the fiber of $p$ over $h\in H$ is the groupoid with objects $(g\in G, p(g)\xrightarrow{\sigma} h)$ and morphisms being $g\to g'$ which make everything commute. Denote the fiber $F_h$, and its canonical map to $G$ by $F_h\xrightarrow{i}G$. 
\end{Definition}
\begin{Definition}
Say that a morphism $G\xrightarrow{p} H$ of groupoids is \emph{finitary} if $F_h$ has finitely many isomorphism classes for every $h$. 
\end{Definition}

We have an obvious functor $p^*:\Rep(H)\to\Rep(G)$ given by precomposition.

\begin{Proposition}
$p^*$ has a bi-adjoint functor $p_*$
\end{Proposition}
\begin{proof}
Let $\rho\in\Rep(G)$ and $h\in H$. We give an explicit formula for $p_*(\rho)(h)$ using the fiber of $p$ over $h$.

Denote by $[F_h]$ the set of isomorphism classes of $F_h$, then the formula for the adjoint is
\begin{equation}
p_*(\rho)(h)=\bigoplus_{x\in [F_h]} \rho(i(x))\otimes_{\field [\Aut_{F_h}(x)]}\field
\end{equation}

(the tensoring with $\field$ is just taking co-invariants with respect to $\Aut_{F_h}(x)$).

To complete the description we need to explain how $p_*(\rho)$ acts on maps in $H$. So let $h\xrightarrow{\tau}h'$ be a map, which since $H$ is a groupoid must be an isomorphism. First, note that $\tau$ induces an equivalence $F_h\to F_{h'}$ by $(g, p(g)\xrightarrow{\sigma} h)\mapsto(g, p(g)\xrightarrow{\tau\circ\sigma} h') $. In particular it induces a bijection $[F_h]\to [F_{h'}]$ which we will also denote by $\tau$. The map $p_*(\rho)(h)\to p_*(\rho)(h')$ is then given by the maps $\rho(i(x))\to\rho(i(\tau(x)))$. It is straightforward to check that this is well defined, and functorial in $\rho$.

Now we need to give the unit and co-unit maps. Let us assume for simplicity that $p$ is faithful, and hence $\Aut_{F_h}(x)$ is trivial for all $x$. Let $\rho\in\Rep(G),\theta\in\Rep(H)$, then
\begin{eqnarray}
p_*p^*(\theta)(h)&=\bigoplus_{x\in [F_h]} \theta(p(i(x)))\cong \bigoplus_{x\in [F_h]} \theta(h)\\
p^*p_*(\rho)(g)&=p_*(\rho)(p(g))=\bigoplus_{x\in [F_{p(g)}]} \rho(i(x))
\end{eqnarray}

The maps $\Id\to p_*p^* \to \Id$ are just the diagonal and averaging maps. The maps $\Id\to p^*p_* \to \Id$ are partial diagonal and averaging maps, using the isomorphisms $i(x)\cong g$ lifting $p(i(x))\cong p(g)$, whenever the lift exists.
\end{proof}