\setcounter{secnumdepth}{3}
%packages
\usepackage{amsmath}
\usepackage{amssymb}
\usepackage{amsthm}
\usepackage{comment}
\usepackage[backend=bibtex,style=alphabetic,citestyle=alphabetic]{biblatex}
\addbibresource{Bibliography.bib}
\usepackage{mathtools}
\usepackage[title,toc]{appendix}
%\usepackage[2cell,ps]{xy}
\usepackage{hyperref}
\usepackage{bbm}
\usepackage[utf8]{inputenc}
\usepackage{tikz-cd}
\usepackage{wasysym}
\usepackage{varwidth}
\usepackage{mathtools}
\usepackage{xspace}
\usepackage{aliascnt}
\usepackage[bbgreekl]{mathbbol}

\makeatletter
\renewcommand*\sectionautorefname{\S\@gobble}
\renewcommand*\subsectionautorefname{\S\@gobble}
\renewcommand*\subsubsectionautorefname{\S\@gobble}
\makeatother

%xymatrix stuff
\usepackage[2cell,ps]{xy}
\input xy
\UseAllTwocells
\xyoption{all}
\newcommand{\cmatrix}[2][]{\vcenter{\xymatrix #1{#2}}}
\newcommand{\smatrix}[1]{\vcenter{\xymatrix @C=0.2cm@R=0.3cm{#1}}}

% %comments in article
 \def\adam#1{
 \begin{tikzpicture}[anchor=base, baseline]%
 \draw (0,0) node[rectangle,%
       rounded corners,%
       shading=axis,%
       left color=green!50!white,%
       right color=green!50!white,%
       draw=red,%
       thick%
       ](A){\begin{varwidth}{30em}#1\end{varwidth}};%
 \end{tikzpicture}
 \cite{comment}%
 }

 \def\lena#1{
 \begin{tikzpicture}[anchor=base, baseline]%
 \draw (0,0) node[rectangle,%
       rounded corners,%
       shading=axis,%
       left color=red!50!white,%
       right color=red!50!white,%
       draw=blue,%
       thick%
       ](A){\begin{varwidth}{30em}#1\end{varwidth}};%
 \end{tikzpicture}%
  \cite{comment}
 }

 \def\markp#1{
 \begin{tikzpicture}[anchor=base, baseline]%
 \draw (0,0) node[rectangle,%
       rounded corners,%
       shading=axis,%
       left color=blue!50!white,%
       right color=blue!50!white,%
       draw=red,%
       thick%
       ](A){\begin{varwidth}{30em}#1\end{varwidth}};%
 \end{tikzpicture}%
  \cite{comment}
 }

% %turn off comments
% %\def\adam#1{}
% %\def\lena#1{}
% %\def\mark#1{}

% math
\newcommand{\Trans}{T}
\mathchardef\mhyphen="2D
\DeclareMathOperator{\Hom}{Hom}
\DeclareMathOperator{\Map}{Map}
\DeclareMathOperator{\pt}{pt}
\DeclareMathOperator{\FinSet}{\mathbf{FinSet}}
\DeclareMathOperator{\Set}{\mathbf{Set}}
\DeclareMathOperator{\POSet}{\mathbf{POSet}}
\DeclareMathOperator{\Cat}{\mathbf{Cat}}
\DeclareMathOperator{\Grpd}{\mathbf{Grpd}}
\DeclareMathOperator{\LinCat}{\mathbf{LinCat}}
\DeclareMathOperator{\Fun}{Fun}
\newcommand{\Func}{\mathcal{F}}
\DeclareMathOperator{\AbCat}{\mathbf{AbCat}}
\DeclareMathOperator{\Ab}{\mathbf{Ab}}
\DeclareMathOperator{\Stacks}{\mathbf{Stacks}}
\DeclareMathOperator{\Ob}{Ob}
\DeclareMathOperator{\Mor}{Mor}
\DeclareMathOperator{\Cub}{Cub}
\DeclareMathOperator{\degen}{degen}
\DeclareMathOperator{\End}{End}
\DeclareMathOperator{\Rep}{Rep}
\newcommand{\Vect}{\mathbf{Vect}}
\newcommand{\AdVect}{\mathbf{Vect_{ad}}}
\newcommand{\CCub}{\mathcal{C}ub}
\DeclareMathOperator{\ind}{Ind}
\DeclareMathOperator{\Bil}{B}
\DeclareMathOperator{\Id}{Id}
\DeclareMathOperator{\image}{Im}
\DeclareMathOperator{\Ker}{Ker}
\DeclareMathOperator{\Psh}{Psh}
\DeclareMathOperator{\Sym}{Sym}
\DeclareMathOperator{\Spaces}{Grpd}
\DeclareMathOperator{\sset}{sSet}
\DeclareMathOperator{\lSet}{lSet}
\DeclareMathOperator{\DLat}{DLat}
\DeclareMathOperator{\Corr}{Corr}
\DeclareMathOperator{\OneCorr}{1Corr}
\DeclareMathOperator{\Coker}{Coker}
\DeclareMathOperator{\Sh}{Sh}
\DeclareMathOperator{\Iso}{Iso}
\DeclareMathOperator{\Ext}{Ext}
\newcommand{\Shh}{\mathcal{S}}
\DeclareMathOperator{\Grid}{Grid}
\DeclareMathOperator{\Sq}{Sq}
\DeclareMathOperator{\Irr}{Irr}
\newcommand{\Gln}{Gl_{n}}
\newcommand{\GL}{GL}
\newcommand{\CCC}{\mathcal{C}}
\newcommand{\Fstar}{\mathcal{F}_*}
\newcommand{\AAA}{\mathcal{A}}
\newcommand{\BBB}{\mathcal{B}}
\newcommand{\III}{\mathbbm{I}}
\newcommand{\Base}{\mathcal{B}}
\newcommand{\FinCo}{\mathcal{FC}}
\newcommand{\FinStar}{\FinSet_*}
\newcommand{\CatStar}{\mathfrak{C}}
\newcommand{\CatAd}{\Cat_{ex}}
\newcommand{\aug}[1]{A(#1)}
\newcommand{\grid}{\operatorname{grid}}
\newcommand{\CatEx}{\Cat_{Ex}}
\newcommand{\FinDisj}{\FinSet^{\sqcup}}
\newcommand{\GSet}{G\operatorname{Set}}
\newcommand{\OrdSet}{\mathbb{\Delta}}
\newcommand{\AugOrdSet}{\mathbb{\Delta}_+}
\newcommand{\OrdDisj}{\OrdSet^{\sqcup}}
\newcommand{\OrdNon}{\OrdSet_+}
\newcommand{\Mon}{\mathbf{Mon}}
\DeclareMathOperator{\CAlg}{CAlg}
\DeclareMathOperator{\PolyFun}{PolyFun}
\newcommand{\Hop}{\mathcal{H}}
\newcommand{\Heis}{\operatorname{Heis}}
\newcommand{\FHeis}{\operatorname{FHeis}}
\newcommand{\XXX}{\mathcal{X}}
\newcommand{\FF}{\mathbbm{F}}
\newcommand{\Ff}{\mathcal{F}}
\newcommand{\NN}{\mathbbm{N}}
\newcommand{\ZZ}{\mathbbm{Z}}
\DeclareMathOperator{\point}{\pt}
\newcommand{\kk}{\mathbbm{k}}
\newcommand{\CC}{\mathbbm{C}}
\newcommand{\QQ}{\mathbbm{Q}}
\newcommand{\PPS}[1]{{\mathcal{P}^{\otimes #1}}}
\newcommand{\CS}[1]{{\mathcal{C}^{\otimes #1}}}
\newcommand{\AAS}[1]{A^{\otimes #1}}
\newcommand{\PP}{\mathcal{P}}
\newcommand{\PPC}[1]{\mathcal{P}^{\times #1}}
\newcommand{\cartesianarrow}{\xrightarrow{+}}
\newcommand{\HH}{\mathcal{H}}
\newcommand{\Dual}{\mathbbm{D}}
\newcommand{\DD}{\mathcal{D}}
\newcommand{\DDD}{\mathcal{D}}
\newcommand{\Double}{\mathcal{D}}
\newcommand{\str}[1]{\operatorname{str}_{#1}}
\newcommand{\TT}{\mathbb{T}}
\newcommand{\Fq}{\mathbb{F}_q}
\newcommand{\AdCat}{\mathbbm{Ad}\operatorname{Cat}}
\newcommand{\kAlg}{\kk\mhyphen\operatorname{alg}}
\newcommand{\Zmod}{\ZZ\mhyphen\operatorname{Mod}}
\newcommand{\TwoVect}{\mathbf{2}\mhyphen \Vect}
\newcommand{\TwoVectAd}{{\mathbbm{2}\mhyphen \Vect}}
\newcommand{\fset}[1]{[#1]}
\newcommand{\pset}[1]{[#1]_*}
\newcommand{\GLn}{GL_n}
\newcommand{\ord}[1]{\left[ #1 \right] }
\newcommand{\ordop}[1]{\left[ #1 \right]_+ }
\newcommand{\brak}[3]{\left#1 \vcenter{#2} \right#3}
\newcommand{\longequal}{=\joinrel=\joinrel=}
\newcommand{\base}[1]{#1_{base}}
\newcommand{\omegaCat}{\omega\mhyphen\Cat}
\newcommand{\smallCat}{\operatorname{sm}\Cat}
\newcommand{\bbox}[1]{{\Box^#1}}
\newcommand{\foldbox}{\Box^f}
\newcommand{\monunit}{\mathbbm{1}}
\newcommand{\field}{\mathbbm{k}}
\newcommand{\indic}{\mathbbm{1}}
\newcommand{\twoarrow}[2]{\arrow[shorten >=1em,shorten <=1em,Rightarrow]{#1}[sloped,above]{#2}}
\newcommand{\triarrow}[5]{\arrow[shorten >=#1cm,shorten <=#2cm,Rightarrow]{#3}[sloped,above,#4]{#5}}
\newcommand{\ee}{\mathbf{e}}
\DeclareMathOperator{\depth}{depth}
\DeclareMathOperator{\Aut}{Aut}
%added by Mark
\newcommand{\GG}{\mathcal{G}}
\newcommand{\RR}{\mathfrak{R}} 
\newcommand{\MMM}{\mathcal{M}}
\newcommand{\EEE}{\mathcal{E}}
\newcommand{\Finpi}{\FinStar^{\rm p.i}}
\newcommand{\git}{\mathbin{
  \mathchoice{/\mkern-6mu/}% \displaystyle
    {/\mkern-6mu/}% \textstyle
    {/\mkern-5mu/}% \scriptstyle
    {/\mkern-5mu/}}}% \scriptscriptstyle
\newcommand{\eps}{\varepsilon}    

%tikzcd stuff
\newcommand{\tik}{\begin{tikzcd}}
\newcommand{\tak}{\end{tikzcd}}

\makeatletter
\def\latearrow#1#2#3#4{%
  \toks@\expandafter{\tikzcd@savedpaths\path[/tikz/commutative diagrams/every arrow,#1]}%
  \global\edef\tikzcd@savedpaths{%
    \the\toks@%
    (\tikzmatrixname-#2)% \noexpand\tikzcd@sourceanchor)%
    to%
    node[/tikz/commutative diagrams/every label] {$#4$}
    (\tikzmatrixname-#3)% \noexpand\tikzcd@targetanchor)
;}}
\makeatother

%scaled tikzcd
\def\stik#1#2{
\begin{tikzpicture}[baseline= (a).base]%
\node[scale=#1] (a) at (0,0){%
\begin{tikzcd}[ampersand replacement=\&]%
#2
\end{tikzcd}%
};%
\end{tikzpicture}
}

\def\widestik#1#2#3#4{
\begin{tikzpicture}[baseline= (a).base]%
\node[scale=#1] (a) at (0,0){%
\begin{tikzcd}[ampersand replacement=\&,column sep=#2em,row sep=#3em]%
#4
\end{tikzcd}%
};%
\end{tikzpicture}
}

\def\nstik#1#2{
\begin{tikzpicture}[baseline= (a).base]%
\node[scale=#1] (a) at (0,0){%
\begin{tikzcd}[ampersand replacement=\&,row sep=small,column sep=large]%
#2
\end{tikzcd}%
};%
\end{tikzpicture}
}

\def\nnstik#1#2{
\begin{tikzpicture}[baseline= (a).base]%
\node[scale=#1] (a) at (0,0){%
\begin{tikzcd}[ampersand replacement=\&,row sep=small,column sep=small]%
#2
\end{tikzcd}%
};%
\end{tikzpicture}
}

%inline tikzcd
\def\ttik#1{
\begin{tikzpicture}[anchor=base, baseline,inner sep=0]%
\node (a) at (0,0){%
\begin{tikzcd}[ampersand replacement=\&]%
#1
\end{tikzcd}%
};%
\end{tikzpicture}
}

\def\smallttik#1{
\begin{tikzpicture}[anchor=base, baseline,inner sep=0]%
\node[scale=0.6] (a) at (0,0){%
\begin{tikzcd}[ampersand replacement=\&]%
#1
\end{tikzcd}%
};%
\end{tikzpicture}
}

\newcommand{\mysetminusD}{\hbox{\tikz{\draw[line width=0.6pt,line cap=round] (3pt,0) -- (0,6pt);}}}
\newcommand{\mysetminusT}{\mysetminusD}
\newcommand{\mysetminusS}{\hbox{\tikz{\draw[line width=0.45pt,line cap=round] (2pt,0) -- (0,4pt);}}}
\newcommand{\mysetminusSS}{\hbox{\tikz{\draw[line width=0.4pt,line cap=round] (1.5pt,0) -- (0,3pt);}}}

\newcommand{\mysetminus}{\mathbin{\mathchoice{\mysetminusD}{\mysetminusT}{\mysetminusS}{\mysetminusSS}}}

\newcommand{\mapstack}[2]{
\begin{tikzpicture}[anchor=base, baseline,inner sep=0, row sep=0]%
\node[scale=0.6] (b) at (0,0.3){
$#1$
};%
\node[scale=0.6] (a) at (0,0){%
$#2$
};%
\end{tikzpicture}
}
\def\smallsquare{
\begin{tikzpicture}[anchor=base, baseline]%
\node[scale=0.2] (a) at (0,0){%
\begin{tikzcd}[ampersand replacement=\&]%
\cdot \ar{r} \ar{d} \& \cdot \ar{d}\\ \cdot \ar{r} \& \cdot
\end{tikzcd}%
};%
\end{tikzpicture}
}

\def\leftadjarrow{leftarrow,dashed,red,thick}

\newcommand{\boldtitle}[1]{\begin{flushleft}\textbf{#1}\end{flushleft}}


%theorems
%\theoremstyle{theorem}
\newtheorem{Theorem}{Theorem}
\def\Theoremautorefname{Theorem}
\newtheorem{Proposition}{Proposition}[section]
\def\Propositionautorefname{Proposition}
\newaliascnt{Corollary}{Proposition}
\newtheorem{Corollary}[Corollary]{Corollary}
\aliascntresetthe{Corollary}
\def\Corollaryautorefname{Corollary}
\newaliascnt{Lemma}{Proposition}
\newtheorem{Lemma}[Lemma]{Lemma}
\aliascntresetthe{Lemma}
\def\Lemmaautorefname{Lemma}
\newtheorem* {Claim}{Claim}
\def\Claimautorefname{Claim}
\newtheorem{Conjecture}{Conjecture}
\def\Conjectureautorefname{Conjecture}

\theoremstyle{definition}
\newaliascnt{Definition}{Proposition}
\newtheorem{Definition}[Definition]{Definition}
\aliascntresetthe{Definition}
\def\Definitionautorefname{Definition}

\theoremstyle{remark}
\newaliascnt{Notation}{Proposition}
\newtheorem{Notation}[Notation]{Notation}
\aliascntresetthe{Notation}
\def\Notationautorefname{Notation}
\newaliascnt{Remark}{Proposition}
\newtheorem{Remark}[Remark]{Remark}
\aliascntresetthe{Remark}
\def\Remarkautorefname{Remark}
\newtheorem* {Note}{Note}
\def\Noteautorefname{Note}
\newaliascnt{Example}{Proposition}
\newtheorem{Example}[Example]{Example}
\aliascntresetthe{Example}
\def\Exampleautorefname{Example}
\newtheorem* {Idea}{Idea} 
\def\Ideaautorefname{Idea}
\newtheorem* {Observation}{Observation}
\def\Observationautorefname{Observation}

% \renewcommand*{\thesection}{\Roman{section}}
% \renewcommand*{\thesubsection}{\thesection.\Alph{subsection}}


\def\equationautorefname~#1\null{%
(#1)\null
}


