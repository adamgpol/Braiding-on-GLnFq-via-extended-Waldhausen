\section{Transfer theories}
In this section we outline two kinds of \emph{transfer theories}: The first  shows how to recover the usual Ringel-Hall algebra from our geometric Hall algebra construction and the second transfers the structure of the geometric Hall algebra to the world of linear categories. In particular using it we
recover the monoidal structure and the braiding on $\bigoplus_n \Rep(\GL(n,\FF_q))$ and \lena{other example?}

\subsection{Transfer to $\Vect$}
\label{VectTransfer}
\lena{Explain how 2-morphisms in $\Corr{\Spaces}$ give equalities}
This construction is based on a functor from the 1-category of correspondences of groupoids to $\Vect$ described in \cite[\S8.2]{KapranovDyckerhoff}. 

A stack $X$ over $\field$ is sent to the vector space of finitely supported functions on the set $\pi_0(X(\field))$ (i.e. the isomorphism classes of $X(\field)$) which we denote $\Func(X)$.

Given a correspondence $X \xleftarrow{s} Y \xrightarrow{p} Z$ we first send it to the correspondence of groupoids $X(\field) \xleftarrow{s} Y(\field) \xrightarrow{p} Z(\field)$ and then to the map $\Func(X) \to \Func(Z)$ given by $p_!s^*$. This assumes some restrictions on $s$ and $p$ (see \cite[\S 2]{Dyckerhoff}) which are always satisfied in the cases we consider. 

It follows immediately from \cite[Proposition 2.17]{Dyckerhoff} that this assignment is a transfer theory to $\Vect$.

Applying this transfer to $H_{geo}$ recovers the usual Hall algebra.
\subsection{Linear categories}
\label{LinCatTransfer}
Let $X$ be a groupoid and define $\Trans(X)$ to be the category of finitely supported representations of the groupoid $X(\FF_q)$ in $\Vect_\CC$. Using the pushforward defined in \autoref{GroupoidRep} we see that $\Trans$ is naturally a functor and for any $f:X\to Y$, $\Trans(f)$ has a biadjoint with explicit adjunctions. This means that for any square of correspondences we can produce a 2-morphism.



