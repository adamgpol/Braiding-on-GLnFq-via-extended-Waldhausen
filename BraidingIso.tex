\section{The Braiding is an isomorphism}
\label{BraidingIso}

%Obviously this title needs to change. The contents of this section should appear in the paper, but the exact way in which it happens depends on broader content aspects.

The goal of this section is to prove that the braiding introduced in Section ***REF*** is in fact an isomorphism. To begin, we reduce the braiding square ***REF*** to a form which is amenable to direct calculation.

Given non-negative integers $n_1, \ldots, n_k$ with $n_1 + \cdots n_k = n$ we define the following subgroups of $\GLn$: 
\begin{itemize}
\item $\GL_{n_1,\ldots,n_k} := \GL_{n_1} \times \cdots \times \GL_{n_k}$, the block diagonal subgroup;
\item  $P_{n_1,\ldots,n_k}$, the parabolic subgroup corresponding to the flag 
\begin{equation*}
    0 \subset \CC^{n_1} \subset \CC^{n_1 + n_2} \subset\cdots \subset \CC^{n_1 + \cdots n_k}.
\end{equation*}
\item $K_{n_1,\ldots, n_k}$, the kernel of the surjection 
\begin{equation*}
\pi_{n_1,\ldots, n_k}:\xymatrix{P_{n_1,\ldots, n_k} \ar@{->>}[r] & \GL_{n_1,\ldots, n_k}}
\end{equation*}
sending a flag-preserving automorphism to the induced automorphism of the relevant quotient spaces. One therefore has that 
\begin{equation*}
    P_{n_1,\ldots, n_k} = K_{n_1,\ldots, n_k} \rtimes \GL_{n_1,\ldots, n_k};
\end{equation*}
 
\item  $\PP_{n_1, n_2, n_3, n_4}$, the intersection of the parabolic subgroups $P_{n_1, n_2, n_3, n_4}$ and $P_{n_1, n_3, n_2, n_4}$.
\end{itemize}

 
\begin{Proposition}
The square defining the braiding ***REF*** is equivalent to the correspondence
\begin{equation}
    \xymatrix{ & \ar@{->>}[ld]_-\pi \coprod_{a,b,c,d} P_{a,b,c,d} \ar@{^{(}->}[rd]^-i & \\
    \coprod_{a,b,c,d} \GL_{a,b,c,d} & \coprod_{a,b,c,d} \PP_{a,b,c,d} \ar@{^{(}->}[u]^-I \ar@{^{(}->}[d]_-{\tilde{I}} & \coprod_{n} \GLn, \\
    & \ar@{->>}[lu]^-{\tilde{\pi}} \coprod_{a,b,c,d} P_{a,c,b,d}\ar@{^{(}->}[ru]_-{\tilde{i}} & }
\end{equation}
where $I$, $i$, $\tilde{I}$ and $\tilde{i}$ are given by the obvious subgroup inclusions, and $\pi$ and $\tilde{\pi}$ are given by the surjections $\pi_{a,b,c,d}$.
\end{Proposition}
\begin{proof}
Composing the outer edges of the square yields the correspondence

****** NICE FIGURE WITH H's, I's AND GRID ******

Therefore it suffices to determine skeletal forms for the groupoid ***I***, ***H*** and ****GRID*** and the induced functors between them. Further, note that the skeletons of each of these groupoids have object set given by $4$-tuples of non-negative integers. We proceed by fixing such a $4$-tuple $a,b,c,d$ and determining the corresponding automorphism groups.

For the groupoid ***I***, the automorphism group is the intersection of the parabolics $P_{a,b,a+b+c+d}$ and $P_{c,c+d} \subset \GL_{c+d} \subset \GL_{a+b+c+d}$ inside $\GL_{a+b+c+d}$. This is exactly the parabolic $P_{a,b,c,d}$. The same argument, {\em mutatis mutandi}, proves the claim for the groupoid ***H***.

Finally, for the groupoid ***GRID***, any automorphism must fix both the ***I***-shaped and ***H***-shaped subdiagrams, so the automorphism group must be a subgroup of $P_{a,b,c,d} \cap P_{a,c,b,d}$. One readily verifies that {\em every} element of this intersection defines an automorphism of the grid. %Not great write-up. 
\end{proof}

The braiding $\beta$ is therefore given by the following composition
\begin{equation}
    \xymatrix{i_* \pi^* \ar@{=>}[r]^-{\eta^R_I} & i_* I_* I^* \pi^* \ar@{=}[r] & \tilde{i}_* \tilde{I}_* \tilde{I}^* \tilde{\pi}^* \ar@{=>}[r]^-{\eps_{\tilde{I}}^L} & \tilde{i}_* \tilde{\pi}^*}.
\end{equation}
To show that $\beta$ is a natural isomorphism it suffices to show that for each $4$-tuple $a,b,c,d$, and each $\rho \in \Rep(\GL_{a,b,c,d})$ the component $\beta(\rho)$ is an isomorphism.



\begin{Lemma}
The components of the braiding $\beta$ corresponding to a representation $\rho \in \Rep(\GL_{a,b,c,d})$ are given by
\begin{align}
\CC[\GL_n]\otimes_{\CC[P_{a,b,c,d}]} \rho &\to \CC[\GL_n] \otimes_{\CC[P_{a,c,b,d}]} \rho \\ \nonumber
\gamma \otimes v &\mapsto \frac{\left|\GL_{a,b,c,d}\right|}{\left| \PP_{a,b,c,d} \right|} \sum_{k \in K_{a,b,c,d}} \gamma k \otimes v. \nonumber
\end{align}
\end{Lemma}
\begin{proof}
By equation \eqref{unitexp}, $\eta^R_I$ sends $\gamma \otimes v$ to the element of $\CC[\GL_n] \otimes_{\CC[P_{a,b,c,d}]} \left( \CC[P_{a,b,c,d}] \otimes_{\CC[\PP_{a,b,c,d}]} \rho\right)$  given by %doesn't look nice
\begin{equation*}
\frac{1}{\left| \PP_{a,b,c,d} \right|} \sum_{p \in P_{a,b,c,d}}\gamma \otimes p \otimes p^{-1} \cdot v = \frac{1}{\left| \PP_{a,b,c,d} \right|} \sum_{p \in P_{a,b,c,d}}\gamma \otimes p \otimes \pi_{a,b,c,d}(p)^{-1} v.
\end{equation*}
Since $P_{a,b,c,d} = K_{a,b,c,d} \rtimes \GL_{a,b,c,d}$ every $p \in P_{a,b,c,d}$ can be uniquely written as $p=k g$ where $k \in K_{a,b,c,d}$ and $g \in \GL_{a,b,c,d}$ where $g = \pi_{a,b,c,d}(p)$. Therefore the image of $\gamma \otimes v$ is equal to
\begin{equation*}
 \frac{1}{\left| \PP_{a,b,c,d} \right|} \sum_{\substack{ k \in K_{a,b,c,d} \\ g \in \GL_{a,b,c,d}
 }}\gamma \otimes k g \otimes g^{-1} v = \frac{\left|\GL_{a,b,c,d}\right|}{\left| \PP_{a,b,c,d} \right|} \sum_{k \in K_{a,b,c,d}} \gamma k \otimes e \otimes v,
\end{equation*}
where we used that $\GL_{a,b,c,d} \subset \PP_{a,b,c,d}$ and $K_{a,b,c,d} \subset P_{a,b,c,d}$.

Applying the counit $\eps^L_{\tilde{I}}$ to the above expression yields by equation \eqref{counitexp} the claimed expression for the components of $\beta$.
\end{proof} 

\begin{Theorem}
The braiding $\beta$ is an isomorphism.
\end{Theorem}
\begin{proof}

\end{proof}